% Copyright 2013 Christophe-Marie Duquesne <chmd@chmd.fr>
% Copyright 2014 Mark Szepieniec <http://github.com/mszep>
%
% ConText style for making a resume with pandoc. Inspired by moderncv.
%
% This CSS document is delivered to you under the CC BY-SA 3.0 License.
% https://creativecommons.org/licenses/by-sa/3.0/deed.en_US

\startmode[*mkii]
  \enableregime[utf-8]
  \setupcolors[state=start]
\stopmode

\setupcolor[hex]
\definecolor[titlegrey][h=757575]
\definecolor[sectioncolor][h=397249]
\definecolor[rulecolor][h=9cb770]

% Enable hyperlinks
\setupinteraction[state=start, color=sectioncolor]

\setuppapersize [A4][A4]
\setuplayout    [width=middle, height=middle,
                 backspace=20mm, cutspace=0mm,
                 topspace=10mm, bottomspace=20mm,
                 header=0mm, footer=0mm]

%\setuppagenumbering[location={footer,center}]

\setupbodyfont[11pt, helvetica]
\usesymbols[fontawesome5]

\setupwhitespace[medium]

\setupblackrules[width=31mm, color=rulecolor]

\setuphead[chapter]      [style=\tfd]
\setuphead[section]      [style=\tfd\bf, color=titlegrey, align=middle]
\setuphead[subsection]   [style=\tfb\bf, color=sectioncolor, align=right,
                          before={\leavevmode\blackrule\hspace}]
\setuphead[subsubsection][style=\bf]

\setuphead[chapter, section, subsection, subsubsection][number=no]

%\setupdescriptions[width=10mm]

\definedescription
  [description]
  [headstyle=bold, style=normal,
   location=hanging, width=18mm, distance=14mm, margin=0cm]

\setupitemize[autointro, packed]    % prevent orphan list intro
\setupitemize[indentnext=no]

\setupfloat[figure][default={here,nonumber}]
\setupfloat[table][default={here,nonumber}]

\setuptables[textwidth=max, HL=none]
\setupxtable[frame=off,option={stretch,width}]

\setupthinrules[width=15em] % width of horizontal rules

\setupdelimitedtext
  [blockquote]
  [before={\setupalign[middle]},
   indentnext=no,
  ]


\starttext

\section[title={Eirik Rolland Enger},reference={eirik-rolland-enger}]

\thinrule

\startblockquote
{\bf PhD candidate}

PhD candidate at the complex systems modelling group at the Department
of Physics and Technology, University of Tromsø. Fond of abstract ideas,
free open-source software and skiing.
\stopblockquote

\thinrule

\subsection[title={Education},reference={education}]

\startdescription{2020--2024 (expected)}
  {\em PhD, Climate Physics at the University of Tromsø} (Tromsø,
  Norway)

  {\em Thesis title: Deep Learning Approaches to the Self-Awesomeness
  Estimation Problem}
\stopdescription

\startdescription{2015--2020}
  {\em MS in Space Physics at the University of Tromsø} (Tromsø, Norway)

  {\em Thesis title: A model for IS spectra for magnetized plasma with
  arbitrary isotropic velocity distributions.} Link:
  \useURL[url1][https://hdl.handle.net/10037/19542]\from[url1]
\stopdescription

\subsection[title={Experience},reference={experience}]

\startdescription{2018--Now}
  {\em Teacing Assistant at University of Tromsø} (Tromsø, Norway).

  \startitemize[packed]
  \item
    FYS-2000 Quantum Mechanics (S18)
  \item
    FYS-0100 Basic Physics (F18,F19)
  \item
    FYS-2009 Sun, planets and space (F20,F21)
  \item
    FYS-3002 Techniques for investigating the near-earth space
    environment (S21)
  \stopitemize
\stopdescription

\startdescription{2019 (2 months)}
  {\em Summer student at FFI --- Norwegian Defence Research
  Establishment} (Kjeller, Norway).

  During eight weeks in the summer of 2019 I worked at the FFI,
  continuing the project on software defined radios from 2018. The goal
  this summer was to be able to do real time spoofing of a GNSS (Global
  Navigation Satellite System) receiver, meaning it should be possible
  for the spoofer to make adjustments to the path the fake signal gives,
  in real time. Multiple open-source projects was used, some of which I
  modified or wrote myself during the project. The added code was
  written in Python, and the complete project can by found in my
  \useURL[url2][https://github.com/engeir/bladeGPS-Game][][bladeGPS-Game
  repository]\from[url2]. The project ended in a successful
  demonstration of real-time controlling of a spoofing signal.
\stopdescription

\startdescription{2018 (3 months)}
  {\em Summer student at FFI --- Norwegian Defence Research
  Establishment} (Kjeller, Norway).

  During nine weeks in the summer of 2018 I worked at the FFI on a
  project about software defined radios for use with jamming and
  spoofing of GNSS receivers. Open-source projects was used along with a
  number of different hardware, most notably the
  \useURL[url3][https://www.ettus.com/all-products/ub210-kit/][][USRP]\from[url3].
  At the end of the period, spoofing of both GNSS receivers and a phone
  was demonstrated, and a report documenting the project was written.
\stopdescription

\subsection[title={Technical
Experience},reference={technical-experience}]

\startdescription{Website}
  I have a website called
  \useURL[url4][https://flottflyt.com][][flottflyt.com]\from[url4] where
  I put up projects I work in my spare time, as well as any other
  content I find interesting. For example I have my own NFT storefront
  on the website that uses the
  \useURL[url5][https://www.metaplex.com/][][metaplex]\from[url5]
  protocol on the
  \useURL[url6][https://solana.com/][][Solana]\from[url6] blockchain.
\stopdescription

\startdescription{Open Source}
  I maintain the project
  {\bf \useURL[url7][https://ncdump-rich.readthedocs.io/][][ncdump-rich]\from[url7]}
  which is published on
  \useURL[url8][https://pypi.org/][][PyPI]\from[url8]. This is a
  previewer for quickly showing formatted metadata in \type{.nc} files,
  written in python. I also contributed to
  {\bf \useURL[url9][https://github.com/Naheel-Azawy/stpv][][stpv]\from[url9]}
  which is a general previewing tool to be used within the terminal.
\stopdescription

\startdescription{Programming Languages}
  {\bf python:} Have been programming in python for four years with
  increasing intensity, creating multiple projects over the years. See
  my \useURL[url10][https://github.com/engeir][][github]\from[url10] for
  a closer look at the different repositories.
\stopdescription

\thinrule

\startblockquote
\useURL[url11][mailto:eirik.r.enger@uit.no][][eirik.r.enger@uit.no]\from[url11]
• +47 477 19 556 • 25 years old\crlf
\useURL[url12][https://eirikenger.xyz][][eirikenger.xyz]\from[url12] •
\useURL[url13][https://github.com/engeir][][github]\from[url13] •
\useURL[url14][https://www.linkedin.com/in/eirik-rolland-enger/][][linkedin]\from[url14]
•
\useURL[url15][https://twitter.com/EngerEirik][][twitter]\from[url15]\crlf
Grenseveien 6, 9011 Tromsø, Norway\crlf
\crlf
\useURL[url16][../output/resume.pdf][][pdf version]\from[url16] •
\useURL[url17][../output/resume.docx][][doc version]\from[url17] •
\useURL[url18][../output/resume.rtf][][rtf version]\from[url18] •
\useURL[url19][../output/resume.html][][html version]\from[url19]
\stopblockquote

\stoptext

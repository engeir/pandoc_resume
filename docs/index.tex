% Copyright 2013 Christophe-Marie Duquesne <chmd@chmd.fr>
% Copyright 2014 Mark Szepieniec <http://github.com/mszep>
%
% ConText style for making a resume with pandoc. Inspired by moderncv.
%
% This CSS document is delivered to you under the CC BY-SA 3.0 License.
% https://creativecommons.org/licenses/by-sa/3.0/deed.en_US

\startmode[*mkii]
  \enableregime[utf-8]
  \setupcolors[state=start]
\stopmode

\setupcolor[hex]
\definecolor[titlegrey][h=757575]
\definecolor[sectioncolor][h=397249]
\definecolor[rulecolor][h=9cb770]

% Enable hyperlinks
\setupinteraction[state=start, color=sectioncolor]

\setuppapersize [A4][A4]
\setuplayout    [width=middle, height=middle,
                 backspace=20mm, cutspace=0mm,
                 topspace=10mm, bottomspace=20mm,
                 header=0mm, footer=0mm]

%\setuppagenumbering[location={footer,center}]

\setupbodyfont[11pt, helvetica]

\setupwhitespace[medium]

\setupblackrules[width=31mm, color=rulecolor]

\setuphead[subsection] [style=\tfd]
\setuphead[chapter]    [style=\tfd\bf, color=titlegrey, align=middle]
\setuphead[section]    [style=\tfb\bf, color=sectioncolor, align=right,
                          before={\leavevmode\blackrule\hspace}]
\setuphead[subsubsection][style=\bf]

\setuphead[chapter, section, subsection, subsubsection][number=no]

%\setupdescriptions[width=10mm]

\definedescription
  [description]
  [headstyle=bold, style=normal,
   location=hanging, width=18mm, distance=14mm, margin=0cm]

\setupitemize[autointro, packed]    % prevent orphan list intro
\setupitemize[indentnext=no]

\defineitemgroup[enumerate]
\setupenumerate[each][fit][itemalign=left,distance=.5em,style={\feature[+][default:tnum]}]

\setupfloat[figure][default={here,nonumber}]
\setupfloat[table][default={here,nonumber}]

\setuptables[textwidth=max, HL=none]
\setupxtable[frame=off,option={stretch,width}]

\setupthinrules[width=15em] % width of horizontal rules

\setupdelimitedtext
  [blockquote]
  [before={\setupalign[middle]},
   indentnext=no,
  ]


\starttext

\startsectionlevel[title={Eirik Rolland
Enger},reference={eirik-rolland-enger}]

\thinrule

\startblockquote
{\bf PhD candidate}

PhD candidate at the complex systems modelling group at the Department
of Physics and Technology, University of Tromsø. Fond of abstract ideas,
free open-source software and skiing.
\stopblockquote

\thinrule

\startsectionlevel[title={Education},reference={education}]

\startdescription{2020--2024 (expected)}
  {\em PhD, Climate Physics at the University of Tromsø} (Tromsø,
  Norway)

  Thesis title: Global temperature response to volcanic activity.

  The PhD work consist of simulating volcanic climate forcing and
  investigate the corresponding temperature response to volcanoes. The
  response to volcanic forcing is hypothesized to be linear. Further,
  analysis is carried out to investigate the universality of the
  response to volcanic forcing with respect to different climate
  forcings, possibly providing valuable insight into the equilibrium
  climate sensitivity.
\stopdescription

\startdescription{2015--2020}
  {\em MS in Space Physics at the University of Tromsø} (Tromsø, Norway)

  Thesis title: A model for IS spectra for magnetized plasma with
  arbitrary isotropic velocity distributions. Link:
  \useURL[url1][https://hdl.handle.net/10037/19542]\from[url1]

  During the Master Thesis work I developed a \goto{python
  program}[url(https://inscar.readthedocs.io/en/latest/)] that solves an
  incoherent scatter radar equation. The equation is solved for any
  oblique angles between the radar pointing direction and the magnetic
  field line, and it accepts any isotropic electron velocity
  distribution. This made it possible to calculate numerically the
  spectrum of super-thermal electrons observed by a moving radar and
  compare this to real observations, which was a new contribution to the
  field.
\stopdescription

\stopsectionlevel

\startsectionlevel[title={Experience},reference={experience}]

\startdescription{2024}
  {\em Course leader at University of Tromsø} (Tromsø, Norway).

  \startitemize[packed]
  \item
    FYS-2019 Sun, planets and space (F24)
  \stopitemize

  Led both the lectures and the exercise classes of FYS-2019 during the
  fall of 2024, giving me valuable practice in preparing course
  material, lecturing and preparation of a 4h written exam.
\stopdescription

\startdescription{2018--2022}
  {\em Teacing Assistant at University of Tromsø} (Tromsø, Norway).

  \startitemize[packed]
  \item
    FYS-2000 Quantum Mechanics (S18)
  \item
    FYS-0100 Basic Physics (F18,F19)
  \item
    FYS-2019 Sun, planets and space (F20,F21)
  \item
    FYS-3002 Techniques for investigating the near-earth space
    environment (S21,S22)
  \stopitemize
\stopdescription

\startdescription{2019 (2 months)}
  {\em Summer student at FFI --- Norwegian Defence Research
  Establishment} (Kjeller, Norway).

  During eight weeks in the summer of 2019 I worked at the FFI,
  continuing the project on software defined radios from 2018. The goal
  was to do real time spoofing of a GNSS (Global Navigation Satellite
  System) receiver, meaning it should be possible for the spoofer to
  make adjustments to the path the fake signal gives, in real time.
  Multiple open-source projects was used, some of which I modified or
  wrote myself during the project. The added code was written in Python,
  and the complete project can by found in my \goto{bladeGPS-Game
  repository}[url(https://github.com/engeir/bladeGPS-Game)]. The project
  ended in a successful demonstration of real-time control of a spoofing
  signal.
\stopdescription

\startdescription{2018 (3 months)}
  {\em Summer student at FFI --- Norwegian Defence Research
  Establishment} (Kjeller, Norway).

  During nine weeks in the summer of 2018 I worked at the FFI on a
  project about software defined radios for use with jamming and
  spoofing of GNSS receivers. Open-source projects was used along with a
  number of different hardware, most notably the
  \goto{USRP}[url(https://www.ettus.com/all-products/ub210-kit/)]. At
  the end of the period, spoofing of both GNSS receivers and a mobile
  phone was demonstrated, and a report documenting the process was
  written.
\stopdescription

\stopsectionlevel

\startsectionlevel[title={Technical
Experience},reference={technical-experience}]

\startdescription{Website}
  I have a website called \goto{eirik.re}[url(https://eirik.re)] where I
  put up projects I work on in my spare time, as well as other content I
  find interesting.
\stopdescription

\startdescription{Open Source}
  Maintainer of the projects {\bf
  \goto{northern-lights-forecast}[url(https://github.com/engeir/northern-lights-forecast)]}
  (archived) and {\bf
  \goto{ncdump-rich}[url(https://ncdump-rich.readthedocs.io/)]} which
  are both published on \goto{PyPI}[url(https://pypi.org/)].
  northern-lights-forecast is a program that listens to a website for
  updates on northern lights weather, and sends a message to a Telegram
  bot if conditions for seeing northern lights are good. ncdump-rich is
  a previewer for quickly showing formatted metadata in \type{.nc}
  files, written in python. Also made contributions to {\bf
  \goto{stpv}[url(https://github.com/Naheel-Azawy/stpv)]} which is a
  general previewing tool to be used within the terminal, for example
  with the file manager
  \goto{lf}[url(https://godoc.org/github.com/gokcehan/lf)].
\stopdescription

\startdescription{Programming Languages}
  {\bf python:} Have been programming in python for four years with
  increasing intensity, creating multiple projects over the years. See
  my \goto{github}[url(https://github.com/engeir)] for a closer look at
  the different repositories. {\bf go:} currently learning go following
  the \goto{interpreterbook}[url(https://interpreterbook.com/)] and the
  \goto{compilerbook}[url(https://compilerbook.com/)].
\stopdescription

\thinrule

\startblockquote
\goto{engeir@pm.me}[url(mailto:engeir@pm.me)] • +47 477 19 556 • 28
years old\crlf
\goto{eirik.re}[url(https://eirik.re)] •
\goto{github}[url(https://github.com/engeir)] •
\goto{linkedin}[url(https://www.linkedin.com/in/eirik-rolland-enger/)] •
\goto{twitter}[url(https://twitter.com/EngerEirik)]\crlf
Elveslettvegen 125, 9020 Tromsdalen, Norway\crlf
Last update: 13 July 2024\crlf

\goto{pdf version}[url(https://resume.eirikenger.xyz/index.pdf)] •
\goto{doc version}[url(https://resume.eirikenger.xyz/index.docx)] •
\goto{rtf version}[url(https://resume.eirikenger.xyz/index.rtf)] •
\goto{html version}[url(https://resume.eirikenger.xyz)]
\stopblockquote

\stopsectionlevel

\stopsectionlevel

\stoptext
